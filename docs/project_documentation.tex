\documentclass[11pt,a4paper]{article}

\usepackage[T1]{fontenc}
\usepackage[utf8]{inputenc}
\usepackage{geometry}
\usepackage{hyperref}
\usepackage{longtable}
\usepackage{array}

\geometry{margin=2.5cm}

\title{Project Arduino A2\\Technical Documentation}
\author{Project Arduino A2 Contributors}
\date{\today}

\begin{document}

\maketitle

\tableofcontents
\newpage

\section{Overview}

Project Arduino A2 is an open-source firmware stack for the Arduino Uno. It aggregates multiple sensors (DHT11, BH1750, GY-GPS6MV2, DS1307 RTC) and an RGB LED actuator while logging environmental data to an SD card. Each hardware feature is wrapped in a small module with a consistent \texttt{init} / \texttt{update} interface, keeping the application loop responsive and the code base easy to extend.

\section{Repository Structure}

\begin{longtable}{>{\ttfamily}p{0.35\linewidth}p{0.6\linewidth}}
\textbf{Path} & \textbf{Description} \\
\hline
platformio.ini & PlatformIO build configuration and dependency declarations. \\
src/main.cpp & Firmware entry point that orchestrates module initialisation and updates. \\
src/actuators/rgb/ & RGB LED module (non-blocking colour cycle). \\
src/sensors/dht/ & DHT11 temperature/humidity module with cached readings. \\
src/sensors/bh1750/ & BH1750 light sensor module with address auto-detection. \\
src/sensors/gps/ & TinyGPS++ based GPS module (GY-GPS6MV2). \\
src/sensors/rtc/ & DS1307 real-time clock (TinyRTC) module. \\
src/storage/sd/ & SD card logger module writing CSV data. \\
docs/project\_documentation.tex & This documentation file. \\
\end{longtable}

\section{Hardware Requirements}

\subsection{Bill of Materials}

\begin{itemize}
  \item Arduino Uno (ATmega328P) or pin-compatible board.
  \item DHT11 temperature/humidity sensor (with 10\,k$\Omega$ pull-up if using the bare sensor).
  \item BH1750 digital light sensor breakout (GY-302 or similar).
  \item TinyRTC breakout (DS1307 RTC with backup battery).
  \item GY-GPS6MV2 GPS module (u-blox NEO-6M).
  \item SD card breakout/shield (3.3\,V logic; uses SPI bus).
  \item Common-anode RGB LED plus three current-limiting resistors.
  \item Breadboard / wiring harness and jumpers.
\end{itemize}

\subsection{Pin Assignments}

\begin{longtable}{>{\ttfamily}p{0.26\linewidth}p{0.18\linewidth}p{0.46\linewidth}}
\textbf{Device} & \textbf{Arduino Pin} & \textbf{Notes} \\
\hline
DHT11 DATA & D2 & Internal pull-up enabled in firmware. \\
BH1750 SDA & A4 & I\textsuperscript{2}C data (ensure pull-up to VCC, typically on breakout). \\
BH1750 SCL & A5 & I\textsuperscript{2}C clock. \\
TinyRTC SDA & A4 & Shares I\textsuperscript{2}C bus with BH1750. \\
TinyRTC SCL & A5 & Shares I\textsuperscript{2}C bus with BH1750. \\
GPS TX & D4 & Feeds SoftwareSerial RX. \\
GPS RX & D3 & Optional (only if sending commands back; use level shifter). \\
RGB LED RED & D5 & PWM capable; common-anode LED expected. \\
RGB LED GREEN & D6 & PWM capable. \\
RGB LED BLUE & D9 & PWM capable. \\
SD CS & D10 & Chip-select for SD card. \\
SD MOSI & D11 & SPI MOSI. \\
SD MISO & D12 & SPI MISO. \\
SD SCK & D13 & SPI clock. \\
VCC / GND & 5\,V / GND & Shared across all modules (GPS may prefer 3.3\,V logic on RX). \\
\end{longtable}

\section{Software Setup}

\subsection{Toolchain}
\begin{enumerate}
  \item Install \href{https://platformio.org/}{PlatformIO IDE} or the PlatformIO Core CLI.
  \item Connect the Arduino Uno via USB; PlatformIO auto-detects the toolchain.
  \item Install a LaTeX distribution if you plan to compile this document.
\end{enumerate}

\subsection{Dependencies}

The following libraries are declared in \texttt{platformio.ini} and retrieved automatically:
\begin{itemize}
  \item \texttt{adafruit/DHT sensor library@{\^}1.4.6}
  \item \texttt{adafruit/Adafruit Unified Sensor@{\^}1.1.15}
  \item \texttt{claws/BH1750@{\^}1.3.0}
  \item \texttt{mikalhart/TinyGPSPlus@{\^}1.1.0}
  \item \texttt{arduino-libraries/SD@{\^}1.2.4}
  \item \texttt{adafruit/RTClib@{\^}2.1.4}
\end{itemize}

\subsection{Build and Upload}

\begin{enumerate}
  \item From the project root run \texttt{pio run} to compile.
  \item Upload with \texttt{pio run -t upload}; use \texttt{--upload-port} if necessary.
  \item Open the serial monitor via \texttt{pio device monitor --baud 9600}.
\end{enumerate}

\section{Module Overview}

\subsection{RGB LED Actuator}

\begin{itemize}
  \item Files: \texttt{src/actuators/rgb/rgbled.*}
  \item PWM on pins D5/D6/D9; assumes common-anode LED (set \texttt{COMMON\_ANODE} accordingly).
  \item Exposes \texttt{rgbInit()} and \texttt{rgbUpdate(now)} for non-blocking colour cycling.
\end{itemize}

\subsection{DHT11 Sensor}

\begin{itemize}
  \item Files: \texttt{src/sensors/dht/dhtsensor.*}
  \item Provides \texttt{dhtInit()}, \texttt{dhtUpdate(now)} plus accessors for the latest reading.
  \item Maintains cached humidity/temperature values for other modules (e.g., SD logging).
\end{itemize}

\subsection{BH1750 Sensor}

\begin{itemize}
  \item Files: \texttt{src/sensors/bh1750/bh1750sensor.*}
  \item Automatically tries both I\textsuperscript{2}C addresses (0x23 and 0x5C) during init.
  \item Logs lux readings every second when the sensor is available.
\end{itemize}

\subsection{GPS Module}

\begin{itemize}
  \item Files: \texttt{src/sensors/gps/gpssensor.*}
  \item Uses \texttt{SoftwareSerial} on D4/D3 and TinyGPS++ to parse NMEA sentences.
  \item Prints satellite count, HDOP, UTC time, and fix status every second.
\end{itemize}

\subsection{RTC Module}

\begin{itemize}
  \item Files: \texttt{src/sensors/rtc/rtcsensor.*}
  \item Talks to the DS1307 on the TinyRTC board to obtain wall-clock timestamps.
  \item Caches the latest \texttt{DateTime} for use in logs; warns if the clock is not running.
\end{itemize}

\subsection{SD Logger}

\begin{itemize}
  \item Files: \texttt{src/storage/sd/sdlogger.*}
  \item Initialises the SD card, creates \texttt{data.csv} with a header if missing, and logs aggregated sensor data once per second.
  \item CSV columns: sample index, seconds since boot, DHT temperature/humidity, BH1750 lux, GPS status (fix, latitude, longitude, satellites, HDOP, speed, altitude) and UTC time.
\end{itemize}

\section{Extending the Project}

\subsection{Adding Modules}

\begin{enumerate}
  \item Place new modules under \texttt{src/sensors/} or \texttt{src/actuators/} (or create another subdomain).
  \item Provide matching header/implementation files exposing \texttt{init} and \texttt{update} hooks.
  \item Keep modules non-blocking; coordinate periodic work via timestamps passed from the main loop.
  \item Register the module in \texttt{src/main.cpp} and declare extra dependencies in \texttt{platformio.ini}.
\end{enumerate}

\subsection{Testing Suggestions}

\begin{itemize}
  \item Use PlatformIO unit tests for logic separable from hardware.
  \item For hardware regression, stream serial output to a host script or log file for later analysis.
  \item When many modules print to Serial, prefix messages (e.g., \texttt{[DHT]} / \texttt{[SD]}) for clarity.
\end{itemize}

\section{Contribution Guidelines}

\begin{itemize}
  \item Follow the directory conventions outlined above.
  \item Document any new wiring requirements in this file and in code comments where necessary.
  \item Keep code comments concise, focusing on non-obvious behaviour or hardware nuances.
  \item Provide test evidence or reproduction steps in pull requests.
\end{itemize}

\section{Licensing}

Select an open-source licence (MIT, Apache-2.0, GPL-3.0, etc.), place the text in a \texttt{LICENSE} file at the repository root, and reference it here once chosen.

\section{Document Compilation}

Build this document with:
\begin{verbatim}
pdflatex docs/project_documentation.tex
\end{verbatim}
Run twice to resolve the table of contents.

\section{Acknowledgements}

This project relies on community-maintained libraries from Adafruit, TinyGPSPlus, and the BH1750 ecosystem. Please consider contributing improvements back to those upstream projects.

\end{document}
